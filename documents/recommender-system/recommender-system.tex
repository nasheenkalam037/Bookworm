\documentclass[10pt,a4paper]{article}
\usepackage[utf8]{inputenc}
\usepackage{graphicx}
\usepackage[left=2.50cm, right=2.50cm, top=2.50cm, bottom=2.50cm]{geometry}
\usepackage[colorlinks]{hyperref}
\usepackage{parskip}

\title{Machine Learning Recommendation System}
\author{Jonathan Shahen}
\begin{document}
\maketitle

\section{Introduction}

``A recommender system is a software that exploits user’s preferences to suggests items (movies, products, songs, events, etc…) to users. It helps users to find what they are looking for and it allows users to discover new interesting never seen items.'' \cite{bellini-howto}

\section{Background}
Machine Learning is a field in computing where given a set of data, produce a function that can approximate some unknown function.
With respect to Recommendation Systems, we are given a dataset of user's reviews of certain products and based on similarities between a given user and all other users in the system we wish to recommend products that they have not rated yet.

The following is typical development cycle:
\begin{enumerate}
    \item Collect Data
    \item Process Data
    \item Randomly split data (60-40 or up to 80-20 split): training set and testing set
    \item Train a new algorithm using only the training set
    \item Test the learned algorithm in the previous step using the data stored in the testing set. Record a metric describing how well the algorithm performed.
    \item Tweak parameters to gain better performance from learned algorithm and go to step 3
    \item Finish when your algorithm performs well enough
\end{enumerate}


\subsection{Collaborative filtering}
``Collaborative filtering (CF) is a technique used by recommender systems.
In the narrower sense, collaborative filtering is a method of making automatic predictions (filtering) about the interests of a user by collecting preferences or taste information from many users (collaborating). The underlying assumption of the collaborative filtering approach is that if a person A has the same opinion as a person B on an issue, A is more likely to have B's opinion on a different issue than that of a randomly chosen person. For example, a collaborative filtering recommendation system for television tastes could make predictions about which television show a user should like given a partial list of that user's tastes (likes or dislikes).'' \cite{wiki:collaborative_filtering}

\subsection{Autoencoder}
``An autoencoder is a type of artificial neural network used to learn efficient data codings in an unsupervised manner. The aim of an autoencoder is to learn a representation (encoding) for a set of data, typically for dimensionality reduction, by training the network to ignore signal “noise.” Along with the reduction side, a reconstructing side is learnt, where the autoencoder tries to generate from the reduced encoding a representation as close as possible to its original input, hence its name.'' \cite{wiki:autoencoder}

\section{Proposed Plan Going Forward}
From reading through \cite{bellini-howto,google-cloud-tensorflow,medium-deep-learning-recommendation,tensorflow-official-example}, it appears that an autoencoder is the best method forward for our Minimum Viable Product.

We can use the walkthrough from \cite{bellini-howto}, and use their code from \url{https://github.com/vitobellini/tfautorec/blob/master/autorec.py} as a template to create a solution to our problem.

This provides us with a nice, quick, and easy to follow way of creating the machine learning algorithm and allows us a baseline to compare other algorithms or parameter tweaks to gain performance.

\nocite{*}
\bibliographystyle{IEEEtran}
\bibliography{machine-learning}


\end{document}